\begin{titlepage}
\null
\vfil
\vskip 60pt
\begin{center}
\begin{LARGE}
A Categorical Programming Language \\
\end{LARGE}
\vskip 4em
\begin{Large}
Tatsuya Hagino \\
\end{Large}
\end{center}
\vfil
\begin{center}
Doctor of Philosophy \\		% name of degree
University of Edinburgh \\
1987 \\      			% year of presentation
\end{center}
\vskip 2em
\null
\end{titlepage}

\thispagestyle{empty}
\null
\vfill
\begin{tabular}{l}
{\bf Author's current address:} \\
\qquad Tatsuya Hagino \\
%\qquad Data Processing Center \\
%\qquad Kyoto University \\
%\qquad Kyoto 606 \\
\qquad Faculty of Environment and Information Studies \\
\qquad Keio University \\
\qquad Endoh 5322, Fujisawa city, Kanagawa \\
\qquad Japan 252-0882 \\
\qquad E-mail: {\tt hagino@sfc.keio.ac.jp} \\
\end{tabular}
\vskip 2em
\null
\clearpage

\chapter*{Abstract}

\write16{Start Abstract}
\begingroup
\def\baselinestretch{1}
\small
\normalsize
\vskip -2em
A theory of data types and a programming language based on category
theory are presented.

Data types play a crucial role in programming.  They enable us to write
programs easily and elegantly.  Various programming languages have been
developed, each of which may use different kinds of data types.
Therefore, it becomes important to organize data types systematically so
that we can understand the relationship between one data type and
another and investigate future directions which lead us to discover
exciting new data types.

There have been several approaches to systematically organize data
types: algebraic specification methods using algebras, domain theory
using complete partially ordered sets and type theory using the connection
between logics and data types.  Here, we use category theory.  Category
theory has proved to be remarkably good at revealing the nature of
mathematical objects, and we use it to understand the true nature
of data types in programming.

We organize data types under a new categorical notion of {\it
$F,G$-dialgebras} which is an extension of the notion of adjunctions as
well as that of $T$-algebras.  $T$-algebras are also used in domain
theory, but while domain theory needs some primitive data types, like
products, to start with, we do not need any.  Products, coproducts and
exponentiations (i.e.\ function spaces) are defined exactly like in
category theory using adjunctions.  $F,G$-dialgebras also enable us to
define the natural number object, the object for finite lists and other
familiar data types in programming.  Furthermore, their symmetry allows
us to have the dual of the natural number object and the object for
infinite lists (or lazy lists).

We also introduce a functional programming language in a categorical
style.  It has no primitive data types nor primitive control structures.
Data types are declared using $F,G$-dialgebras and each data type is
associated with its own control structure.  For example, natural numbers
are associated with primitive recursion.  We define the meaning of the
language operationally by giving a set of reduction rules.  We also
prove that any computation in this programming language terminates using
Tait's computability method.

A specification language to describe categories is also included.  It is
used to give a formal semantics to $F,G$-dialgebras as well as to give a
basis to the categorical programming language we introduce.
\par
\endgroup
\normalsize
\write16{End Abstract}

\chapter*{Acknowledgements}

The greatest thanks go to Professor Rod Burstall who, first of all,
accepted me as a Ph.\ D. student, then, supervised me during my Ph.\ D.
study and, further, gave me an opportunity to continue working in
Edinburgh.  He gave me not only useful advices but also much-needed
mental support.  I would also thank to his wife, Sissi Burstall, who
invited me to their house often and made my stay in Edinburgh very
pleasant.

I am also grateful to Professor Reiji Nakajima for having encouraged me
to come to Edinburgh and to Professor Heisuke Hironaka for having helped
me to solve the financial problem for studying in Edinburgh.

Many lectures I attended in the first year enlightened me a lot: domain
theory, operational semantics, denotational semantics, algebraic
specification, category theory, and so on.  I did not know what category
theory really is before I came to Edinburgh.  Thanks to Andrzej
Tarlecki, Edmund Robinson and John Cartmell for having helped me to
overcome the initial difficulty of category theory.  John Cartmell also
helped me a lot through discussions in the early stage of the thesis.  I
am also in debt to Furio Honsell who introduced me to Tait's method for
proving normalization theorems of lambda calculi when I stuck in the
normalization proof of CPL.  Many other people in Edinburgh helped me
with their comments and through discussions.  I would especially like to
thank Bob McKay and Paul Taylor for reading the early drafts and
discovering some disastrous mistakes.

This thesis is written using \LaTeX\ on a Sun workstation and printed by
an Apple LaserWriter.  I would like to thank George Cleland and Hugh
Stabler for providing wonderful computing facilities and software to
Laboratory of Foundation of Computer Science.

My Ph.\ D. study at the University of Edinburgh has been funded by the
Educational Project for Japanese Mathematical Scientists, Harvard
University, by the Overseas Research Students Award and by the Science
and Engineering Research Council Research Fellowship.

\tableofcontents

