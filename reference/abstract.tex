\begingroup
\def\baselinestretch{1}
\small
\normalsize
A theory of data types based on category theory is presented.

Data types play a crucial role in programming.  They enable us to write
programs easily and elegantly.  Various programming languages have been
developed, each of which may use different kinds of data types from the
others.  Therefore, it becomes important to organize data types
systematically so that we can understand the relationship between one
data type and another and the reason why we can use those data types but
not any others.  Once we get a clear systematic view of data types, it
may become possible to predict the future direction to find more
exciting data types.

There have been several approaches to systematically organize data
types: algebraic specification methods using algebras, domain theory
using complete partially ordered sets and type theory using the connection
between logics and data types.  Here, we use category theory.  Category
theory has proved to be remarkably good at revealing the nature of
mathematical objects, and we use it to understand the true nature
of data types in programming.

We organize data types under a new categorical notion of {\it
$F,G$-dialgebras} which is an extension of the notion of adjunctions as
well as that of $T$-algebras.  $T$-algebras are also used in domain
theory, but while domain theory needs some primitive data types, like
products, to start with, we do not need any.  Products, coproducts and
exponentiations (i.e.\ function spaces) are defined exactly like in
category theory using adjunctions.  $F,G$-dialgebras also enable us to
define the natural number object, the object for finite lists and other
familiar data types in programming.  Furthermore, their symmetry allows
us to have the dual of the natural number object and the object for
infinite lists (or lazy lists).

We also introduce a programming language in a categorical style using
$F,G$-dialgebras as its data type declaration mechanism.  We define the
meaning of the language operationally and prove that any program
terminates using Tait's computability method.

A specification language to describe categories is also included.  It is
used to give a formal semantics to $F,G$-dialgebras as well as to give a
basis to the categorical programming language we introduce.
\par
\endgroup
\normalsize
